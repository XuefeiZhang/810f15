\documentclass[portrait,11pt]{seminar}

\slidefontsizes{10}

\newcommand\bs{\begin{slide*}}
\newcommand\es{\end{slide*}}

\newcommand\bi{\begin{myitemize}}
\newcommand\ei{\end{myitemize}}


\usepackage{ulem}
\usepackage{amsmath,amssymb,amsfonts,amsthm,graphicx}

\usepackage{color,semcolor}
\definecolor{green}{rgb}{0,0.8,0.2}

\newcommand\prob{\mathbb{P}}
\newcommand\E{\mathbb{E}}
\newcommand\SampleSpace{\mathbb{S}}
\newcommand\R{\mathbb{R}}
\newcommand\Z{\mathbb{Z}}
\newcommand\Var{\mathrm{Var}}
\newcommand\Cov{\mathrm{Cov}}
%\newcommand\mydefinition[1]{{\ \uwave{#1}}}
%\newcommand\mydefinition[1]{{\red \textbf{#1}}}
\newcommand\mydefinition[1]{{\textbf{#1}}}
\newcommand\mymath{\blue }
%\newcommand\myproof{\underline{Proof:} }
\newcommand\myproof{{Proof:} }
\newcommand\equals{{=}\,}
\newcommand\given{{\, | \,}}

\newcommand\hd[1]{\centerline{\large\bf #1}}
\newcommand\shd[1]{\underline{\large #1}}

\slideframe{none}
\newenvironment {myitemize} {
                 \begin{list}{$\bullet$ \hfill}
                 {\setlength{\labelwidth}{0.3 cm}
                  %\setlength{\leftmargin}{0em}
                  \setlength{\leftmargin}{0.15cm}
                  \setlength{\itemindent}{0.15cm}
                  \setlength{\labelsep}{0cm}
                  \setlength{\parsep}{0.2 ex}
%                  \setlength{\itemsep}{0.25 cm}
                  \setlength{\itemsep}{0.15 cm}
      \setlength{\topsep}{0.1cm}}} %space between title and 1st item
   {\end{list}}

\newenvironment {myequation} {\vspace{-1mm}\begin{equation*}}{\end{equation*}\vspace{1mm}}

\newenvironment {myeqnarray} {\vspace{-1mm}\mymath \begin{eqnarray*}}{\end{eqnarray*}\vspace{-1mm}}




\begin{document}

\bs {\bf Here is a comment that made part of a response and is worth discussion:}

{\it ``One thing that bothers me about the RCR seminars that I have attended during the last five years is that they mostly have a prohibitive tone. If we encourage students or young researchers to accept their responsibility in doing research by giving famous examples in history of science, I guess we see more results than by showing the catastrophic consequences of misconducting in research.''}

\medskip

{\bf Here is a related concern of mine:}

{\it ``Non-democratic countries often have compulsory courses on political ideology. What is the difference and similarity between requiring a course on RCRS versus a course on, say, Marxist principles.''}

\es

\bs
{\bf Why do we require RCRS training in the first semester for new PhD students?}

\medskip

Perhaps, to make sure that everyone understands the basic definitions of conflicted interests; plagiarism; negligence; reproducibility. 

For example, the following response to the last question:

{\it ``I am more concerned about the milder misconducts. A lot of researchers do not know what the milder misconduct are. They would offend the milder misconduct without intention.''}

But that raises the question: do these milder misconducts actually hurt science?
 
\es

\bs {\bf Do you think the authors (RNSD) did anything wrong? What, if anything, should be done to correct the situation? }

{\bf A}. {\it ``In this situation, I think it is a self-plagiarism case. The published paper is supposed to be original and should be not be an identical paper with something published already. One solution is to retract the article from Nature.''}


{\bf B}. {\it ``I don’t think the authors did anything particularly wrong. I don’t really understand what they have to gain from publishing results many times in different journals. Yes, their publication numbers increase, but because they are on the same thing, they do not have more impactful publications.''}

% 2014 A. ``After reading the blog, I actually do not feel like the authors did anything wrong.''

% 2014 B. ``By publishing almost identical work in two places, they wasted others time and resources... but beyond that I don’t think their paper should be retracted.''

% 2014 C. ``Yes, they published the same result twice. Their paper should be withdrawn from Nature Biotechnology and they should receive due punishment.''

\es
\bs
{\bf Comment on any details of the case that help you to make up your mind about it.}

% 2014 \it ``... they should have cited the original when republishing in order to demonstrate to the reader that the material is not original.''

{\it ``Since any authors expressly agree to avoid submitting manuscripts under peer-review to other journals, P sees the wrong-doing because the chapter was `refereed exactly like a journal publication.' However I doubt that this was the case. It is not uncommon to see journal articles later represented as part of a peer-edited, multi-author work. Some journal copyright agreements allow for this form of reproduction after acceptance.''}

%\medskip

\es

\bs
\it 
%``I believe that they did do something wrong, but that their offense doesn’t constitute research misconduct. By publishing almost identical work in two places, they wasted others time and resources, especially given that their second paper and book chapter were very nearly a replica of their first paper. At this point I think what's done is done. Journals should be careful to check that research isn’t published multiple times, but beyond that I don’t think their paper should be retracted.''

\es

\bs

{\bf Wikipedia}: {\it ``Plagiarism is the `wrongful appropriation' and `stealing and publication' of another author's `language, thoughts, ideas, or expressions' and the representation of them as one's own original work. The idea remains problematic with unclear definitions and unclear rules. The modern concept of plagiarism as immoral and originality as an ideal emerged in Europe only in the 18th century, particularly with the Romantic movement.''
}

This suggests that self-plagiarism is not plagiarism. In multiple-author papers that distinction gets blurred.

Do you think the modern trend toward strong moral and legal defence of intellectual property is helpful for society?

\es

\bs {\bf What moral arguments, if any, are there against self-plagiarism?}

\vspace{3cm}

{\bf What legal arguments, if any, are there against self-plagiarism?}

\es

\bs
{\bf Is intent to deceive required for plagiarism and/or self-plagiarism?}

\es

\bs
{\bf Should a responsible researcher attempt to avoid these RCRS gray areas? How? What are the advantages and disadvantages of following RCRS practices that are not currently universally adopted?}

{\it ``Proper attribution when copying or adapting homework or exam problems may lead to more problems than it solves. In particular, if students are aware that a problem came from a particular source, they may be tempted to find a solutions manual. To avoid this problem, attribution may be provided when the instructor distributes solutions to the homework/exam.''}

\medskip
I HAD NOT THOUGHT OF THIS NICE SOLUTION BEFORE. NEXT TIME I TEACH, I'LL DO IT!

\medskip
SCIENTIFIC NORMS FOR PRINCIPLED CONDUCT ARE ADVANCING QUITE QUICKLY. IN TEN YEARS, THIS MAY BE COMMON PRACTICE.

\es
\bs

{\it ``Citation is your friend. By properly citing figures, problems, and sources of information, authors can avail themselves of calls of plagiarism and point readers to additional resources.''}

\medskip

{\it ``I think researchers should ensure they do not risk any gray areas early in their career, before they have a better sense of what is and is not acceptable. As they get to know the common practices better they will be able to better figure out what people find acceptable and unacceptable.''}

\es
\bs

{\it ``I think that reusing homework and exam problems are not a significant RCRS breach, as there is generally an expectation that they are there for students to use as practice and are not for public consumption.''}

\medskip

{\it ``Yes, the researcher could adapt the problems in his own way (rephrasing it, change part of the problem, give credit to or get permission from textbook authors), and find the data source and make the figures by himself.''}

% 2014 \it ``Responsible researchers should attemp to avoid above behaviors. Creating new problems for homework and exams is more difficult and sometimes nearly impossible for some courses, however, we can always give the source of problems or figures that we are using in talks and courses. People can save time and could not put too much attention to the details by conducting above behaviors. However, negligence in informal situation can trigger a misconduct in formal/official surrounding.''
\es

\bs
% 2014 \it ``It never hurts to cite sources, but if that was done on homework problems, students could look up answers in the original textbook. Following RCRS practices ensure you will not have to worry about plagiarism, but it takes time and effort.''


% 2014 ``Following RCRS practices that are not currently universal can help a researcher to be more serious with his/her work and contribute to the progress of academia society in the long run. But it takes time and effort which can be a difficult thing in practice.''

\es

\bs
% 2014 \it ``Disadvantages may include that since not everybody is following them at the moment , it might create a strange relationship among colleagues.

\es


\bs

{\bf Are there any forms of inappropriate scientific conduct that you think have the combination of severity and prevalence to threaten the proper functioning of modern science? Are you more concerned about the total effect of serious (and presumably rare) misconduct, or milder (and potentially more common) misconduct?}

% 2014 \it ``I am personally concerned about the impact of milder but more common misconduct to the serious misconduct. For serious misconducts, everyone has some sense of consciousness to avoid them, however, for the common misconducts, many people do not have it. And, in my opinion, this can leads to confusion for the extent of what is adopted behavior and what is not, causing more serious misconduct.''


{\it ``I think mixing of economic interests and research funding can risk disrupting the normal functioning of modern research. Specific examples would be research into the effects of fracking, or the effects of releasing different fertilizers into the environment. Here the costs of doing the research is very high, and the scope very small, so it is hard to find people interested in funding the research who do not have an economic interest, and specifically an interest in a report leaning a specific way. Even if the researchers want to do unbiased research, having an interested party so involved with the research can make it hard to remain unbiased.''}


{\it ``I am more concerned about the total effect of milder and potentially more common misconduct. If it is a mild misconduct, people may do it multiple times since they think it does not have serious impact on community of science. Also other researches may not be able to notice those misconduct since it is mild actions.''}

\es
\bs
% 2014 \it ``I think I am more concerned about the effect of serious misconduct. And this is because I believe that it is hard to obey every RCRS rule. People may make some minor mistakes and cause milder misconduct. But if people misconduct severely, I will think they are doing this on purpose and this is unforgivable.''


{\it ``I am worried that in data-driven sciences, reproducibility would cause great problems for the community. Some conclusions are drawn without clearly stating the premises on which the experiments are based. Worse still some researchers choose to selectively report experiments or data to strengthen their conclusions. Somewhat less obvious but equally dangerous, however, is the practice to perform the experiments first and then reverse-engineer a ‘hypothesis’ that is in agreement of the results, or in a more convoluted form, going back and forth a few times. The latter may be considered milder but I doubt that they are rare.''}

\es
\end{document}
