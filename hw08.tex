\documentclass[12pt]{article}
\usepackage{fullpage,url}\setlength{\parskip}{3mm}\setlength{\parindent}{0mm}
\begin{document}

\begin{center}\bf
Homework 8. Due by 5pm on Wednesday 10/28.

Encouraging responsible conduct in class, as teacher and student.

\end{center}
Pages 15--23 of {\em On Being a Scientist}, which were assigned for homework~4, focused on research misconduct.  Related issues arise in the classroom, where students and teachers together built an intellectual climate for the class. Write brief answers to the following questions, by editing the tex file at \url{dept.stat.lsa.umich.edu/~ionides/810/hw08.tex}, and send me the resulting pdf file. 

\begin{enumerate}

\item As a student in a class, what do you think you would do if you had to deal with the following RCRS situations:

(a) You see two classmates helping each other on a homework that was supposed to be carried out independently.

YOUR ANSWER HERE.

(b) You are fairly sure that the classmate on your right in an in-class test is copying from the student on their right.

YOUR ANSWER HERE.

(c) As an undergraduate, a friend asks to borrow your homework. You know that your friend considers he/she needs an A to get in to medical school. The friend is reasonably capable in this class, but has not left enough time to do this assignment before it is due.

YOUR ANSWER HERE.

\item It is sometimes asserted that ``those who cheat only hurt themselves.'' Explain to what extent you agree with this statement.

YOUR ANSWER HERE.

\item Suppose that, while grading homework as a GSI, you suspect that a student has used material from the internet inappropriately in their homework. What would you do?

YOUR ANSWER HERE.

\item As a professor, some options to deal with suspected academic misconduct are as follows:

(A) Do nothing. You see that the students involved are doing very poorly in the class, and they will get their eventual reward anyhow with a poor grade. Besides, you are less than 100\% sure about your suspicions, and it would be bad for all concerned to make an accusation that turned out to be false.

(B) Give a warning, but take no punitive action. Acknowledging that the students who are cheating are stressed by academic pressures and may not have adequate ethical training, you give the whole class a warning to clarify the situation, but take no individual action. This also deals with your concerns about being less than  100\% sure about your suspicions.

(C) Let the student(s) know you suspect misconduct and tell them they will score zero for this assignment. Let them know they will fail the course if this happens again.

(D) Write up a description of your suspicions and turn it in to the office of the Office of the Assistant Dean for Undergraduate Education. The student(s) will go through the formal process described at
\url{http://www.lsa.umich.edu/academicintegrity/students/}

What are the strengths and weaknesses of these approaches? For each approach, either say that it should never be used or give an example of a situation for which that approach is appropriate.

YOUR ANSWER HERE.

\item Beyond dealing with misconduct, sometimes such issues can be avoided by changing the structure of the class. Perhaps teachers can run the class in ways that make cheating less possible.  Suggest one feature a class might have that encourages misconduct (but might have some other academic benefit) and another that discourages misconduct (but might have some other academic cost).

YOUR ANSWER HERE.

\end{enumerate}
\end{document}
