\documentclass[portrait,11pt]{seminar}

\slidefontsizes{10}

\newcommand\bs{\begin{slide*}}
\newcommand\es{\end{slide*}}

\newcommand\bi{\begin{myitemize}}
\newcommand\ei{\end{myitemize}}


\usepackage{ulem}
\usepackage{amsmath,amssymb,amsfonts,amsthm,graphicx}

\usepackage{color,semcolor}
\definecolor{green}{rgb}{0,0.8,0.2}

\newcommand\prob{\mathbb{P}}
\newcommand\E{\mathbb{E}}
\newcommand\SampleSpace{\mathbb{S}}
\newcommand\R{\mathbb{R}}
\newcommand\Z{\mathbb{Z}}
\newcommand\Var{\mathrm{Var}}
\newcommand\Cov{\mathrm{Cov}}
%\newcommand\mydefinition[1]{{\ \uwave{#1}}}
%\newcommand\mydefinition[1]{{\red \textbf{#1}}}
\newcommand\mydefinition[1]{{\textbf{#1}}}
\newcommand\mymath{\blue }
%\newcommand\myproof{\underline{Proof:} }
\newcommand\myproof{{Proof:} }
\newcommand\equals{{=}\,}
\newcommand\given{{\, | \,}}

\newcommand\hd[1]{\centerline{\large\bf #1}}
\newcommand\shd[1]{\underline{\large #1}}

\slideframe{none}
\newenvironment {myitemize} {
                 \begin{list}{$\bullet$ \hfill}
                 {\setlength{\labelwidth}{0.3 cm}
                  %\setlength{\leftmargin}{0em}
                  \setlength{\leftmargin}{0.15cm}
                  \setlength{\itemindent}{0.15cm}
                  \setlength{\labelsep}{0cm}
                  \setlength{\parsep}{0.2 ex}
%                  \setlength{\itemsep}{0.25 cm}
                  \setlength{\itemsep}{0.15 cm}
      \setlength{\topsep}{0.1cm}}} %space between title and 1st item
   {\end{list}}

\newenvironment {myequation} {\vspace{-1mm}\begin{equation*}}{\end{equation*}\vspace{1mm}}

\newenvironment {myeqnarray} {\vspace{-1mm}\mymath \begin{eqnarray*}}{\end{eqnarray*}\vspace{-1mm}}



\usepackage{url}

\begin{document}
\bs 

{\bf
You see two classmates helping each other on a homework that was supposed to be carried out independently.
}

{ \it ``Before I took the Stats 810 class, I'd like to ignore it, doing nothing. Now, I would like to warn them, reminding them that the homework of this class should be carried out independently.''}


% 2014 ``I should email the situation to the GSI, but in real life I will do nothing.''

% 2014 ``I wouldn't like it because it's unfair to other students. I probably wouldn't report them though.''

% 2014 ``Often talking about a problem helps one to clear important concects related to the problem. The homework is given to students in order to help them develop these concepts. So, I see no harm in discussing the problem, as long as one is not directly copying from the other.''


\es
\bs

{\bf 
You are fairly sure that the classmate on your right in an in-class test is copying from the student on their right.
}

{ \it ``Confront them by asking “are you copying from that student?''}

{\it ``Perhaps indicate on my exam what was happening, or let the professor know after the exam.''}

% 2014 ``Well, in this situation, I will tell the professor after the exam.''


% 2014 ``I would concentrate on my own examination. May be talk to him later about it.''

% 2014 ``I think I might done nothing if I am not 100\% sure that they are copying.''

\es

\bs

{\bf
As an undergraduate, a friend asks to borrow your homework. You know that your friend considers he/she needs an A to get in to medical school. The friend is reasonably capable in this class, but has not left enough time to do this assignment before it is due.
}

{\bf A}. {\it ``I would probably let him/her copy the assignment but also make sure that he has understood the work of the assignment. i would try to persuade him to work on the assignment again later on his own and point out the mistakes he has made.''}

\medskip 

{\bf B}. {\it ``Tell them too bad.''}



\es
\bs

{\bf
As an undergraduate, a friend asks to borrow your homework. You know that your friend considers he/she needs an A to get in to medical school. The friend is reasonably capable in this class, but has not left enough time to do this assignment before it is due.
}
\medskip

NOW, SUPPOSE THAT THE MATERIAL IN THE CLASS IS NOT NECESSARY FOR THE FRIEND'S FUTURE COURSES, THE PROFESSOR MAKES THE COURSE BORING, AND IT SEEMS STUPID THAT THIS BUREAUCRATIC REQUIREMENT COULD POTENTIALLY ELIMINATE YOUR FRIEND FROM GETTING IN TO A GOOD MEDICAL SCHOOL. 

\medskip

IN SUCH SITUATIONS, IT CAN SEEM INHUMAN TO REFUSE HELP TO YOUR FRIENDS. IS IT REASONABLE TO HOPE THAT A GOOD FRIEND WOULD NOT ASK FOR THIS FAVOR? 


% 2014 ``Honestly, if they were a good friend and were in a bind, I would probably just let them copy it. If the problem persisted, then I would say something about being being uncomfortable with the copying.''


% 2014 ``I would not give the homework to the student. If the assignment allows for collaboration, I would provide hints and suggestions on how to go about with the homework assignments.''

% 2014 ``I will lend him/her. But I will tell him/her that this is just for reference. He/she should come up with his/her own homework instead of copying''

\es
\bs
{\bf As an undergraduate, a friend asks to borrow your homework. You know that your friend considers he/she needs an A to get in to medical school. The friend is reasonably capable in this class, but has not left enough time to do this assignment before it is due.}

{\it ``If I'm being completely honest, this is something I've more or less been on both sides of. If I trust that my friend won't verbatim copy me, and I know that he generally knows what he’s doing, I'd give him my homework. It's an understandable situation, and one homework assignment is likely not a huge component of the grade. But is this course of action RCRS approved? Certainly not. So I'd have my qualms, but ultimately I'd still give him the homework.''}

\es
\bs
{\bf Churchill}: Madam, would you sleep with me for five million pounds?\\
{\bf Socialite}: My goodness, Mr. Churchill. Well, I suppose we would have to discuss terms, of course.\\
{\bf Churchill}: Would you sleep with me for five pounds?\\
{\bf Socialite}: Mr. Churchill, what kind of woman do you think I am?!\\
{\bf Churchill}: Madam, we've already established that. Now we are haggling about the price.

\medskip

(This is a very old joke where the participants vary dramatically from each telling. It's very unlikely though not impossible that the joke originated from Churchill.) \url{http://www.barrypopik.com}
\es
\bs
{\bf 
It is sometimes asserted that ``those who cheat only hurt themselves.'' Explain to what extent you agree with this statement.
}

\it

% 2014 ``I absolutely agree. Getting a good grade is never be the ultimate goal of test. It is a way to help us better master and review the knowledge.''

{\it ``In the long run, it would be nice if cheaters mainly harm themselves. But in a class setting, if students are graded on a relative scale, then cheaters can artificially reduce the score of other students.''}

% ``I agree completely with this statement. Though there are situations where cheating may be temporarily beneficial (like in 1(c)), however, if action is not taken to learn the material that you cheated on, you will be hurt in the long run.''

% 2014 ``It's certainly true that cheaters hurt themselves, but it hurts everyone else too. It devalues the work that they've done relative to the cheater(s). If you consider learning to be the only purpose of education, then maybe the statement is true. Most of us cannot divest ourselves of grades and whatnot, at which point, cheaters hurt us too.''
\es

\bs
{\it ``I think there are 2 ways this can be untrue. First is sometimes cheating really does not hurt you. If you are forced to take a class but it is not going to be important for you to actually know the material afterwards, then cheating (while maybe morally wrong) is unlikely to really hurt you unless you get caught. Second if the class is curved in any way cheating hurts everyone else in the class by raising the curve. I think the statement is much more true in grad school, where we actually need to know and understand the material not to take an exam, but to be able to effectively do research.''}

\es

\bs

The Washington Post,

\url{http://www.washingtonpost.com/wp-dyn/articles/A57836-2004Sep2.html}

{\it
``The argument that cheaters hurt only themselves is false. Cheaters do hurt other people, and they do so to help themselves. Students cheat because it works. They get better grades and more advantages with less effort. Honest students lose grades, scholarships, recommendations and admission to advanced programs. Honest students must create enough peer pressure to dissuade potential cheaters. Ultimately, students must be willing to step forward and confront those who engage in academic dishonesty.''
}

\es
\bs

Grades are not the only thing that matters, but they do matter.

If someone cheats to get from $B+$ to $A$, then someone else who would have gotten an $A$ gets $A-$ and someone who would have gotten $A-$ gets $B+$.

\medskip

Why do (and don't) grades matter in a PhD program?

\es

\bs

{\bf 
Suppose that, while grading homework as a GSI, you suspect that a student has used material from the internet inappropriately in their homework. What would you do?
}

% 2014  ``Tell Dr.~Jackie Miller and let her take care of it.''

{\it ``I will discuss this matter with the instructor. I think it is the instructor’s responsibility to deal with this situation.''}

\es
\bs

As a professor, some options to deal with suspected academic misconduct are as follows:

(A) Do nothing. You see that the students involved are doing very poorly in the class, and they will get their eventual reward anyhow with a poor grade. Besides, you are less than 100\% sure about your suspicions, and it would be bad for all concerned to make an accusation that turned out to be false.

(B) Give a warning, but take no punitive action. Acknowledging that the students who are cheating are stressed by academic pressures and may not have adequate ethical training, you give the whole class a warning to clarify the situation, but take no individual action. This also deals with your concerns about being less than  100\% sure about your suspicions.

\es
\bs

(C) Let the student(s) know you suspect misconduct and tell them they will score zero for this assignment. Let them know they will fail the course if this happens again.

(D) Write up a description of your suspicions and turn it in to the office of the Office of the Assistant Dean for Undergraduate Education. The student(s) will go through the formal process described at
\url{http://www.lsa.umich.edu/academicintegrity/students/}

\es

\bs

{\it ``Situation (D) should be reserved for severe cases, like cheating on an exam or other blatant misconduct. Referral to the Assistant Dean can have major consequences for a student, so you should be quite sure in your accusation. This action may not be appropriate for smaller offenses, as it is quite severe.''}

\medskip

A COMMON SENTIMENT. BUT, WILL THE ASSISTANT DEAN GIVE MAJOR CONSEQUENCE FOR A RELATIVELY MINOR FIRST OFFENCE? 

\medskip

ALSO, THE OFFICIAL REPORT MAY EXONERATE THE STUDENT. I SUPPOSE THAT, IF THE PROFESSOR WANTS TO TAKE JUSTICE INTO HIS/HER OWN HANDS, THE STUDENT SHOULD BE GIVEN AN OPTION OF ASKING FOR A FORMAL MISCONDUCT JUDGEMENT.

\es

\bs

{\it ``A) should never be used, because B) is more effective and also doesn’t carry damages if your suspicions are incorrect. D) is obviously what should be done if you confident in your suspicions. C) is less work than D) but may potentially create danger if you are incorrect in your suspicions and gave a 0 for nothing.''}

\es

\bs
\it
% 2014 ``I will choose (C). In my opinion, (D) is too harsh for a student which may ruin the life of student, while (A) and (B) are totally useless in preventing misconduct.''


% 2014 ``(A, C, D) should never be used. I think (B) is a good way to either prevent or stop academic misconducts. If we wrong a student without academic misconduct simply because we are suspecting him with no strong evidence, it will hurt the student and affect the student’s learning motivation. (B) solves this potential problem.''



%``If, you are completely confident that students are cheating and the assignment is a major part of the course, then D may be the proper action.''


%What are the strengths and weaknesses of these approaches? For each approach, either say that it should never be used or give an example of a situation for which that approach is appropriate.

\es

%\bs

% 2014 ``A. This approach should never be used. B. This approach would be good if the misconduct is minor, such as copying someone’s homework. It should not be used for major violations such as having another person take an exam for a particular student. C. This approach should never be used. It is better for the professor to report the issue to the Academic Integrity office so they can deal with the sanctions instead of the professor themselves. D. This approach should be used when someone commits a major misconduct such as cheating on exams.''


%{\bf It is my belief that the University does not create serious, career-destroying consequences for those found guilty of relatively minor, first-time infringements.}


%\es

\bs
{\it
% 2014 ``I think approach (A) is not a good idea. A teacher is also responsible for prevention of misconduct. Nor do I think option (D) should be resorted to unless under extreme circumstances. Approach (B) may be used if you are not absolutely convinced whether or not misconduct is indeed taking place. In general I would prefer approach (C), because it is not too extreme, however,it lets the students know that you are aware exactly what they are up to.''
}

\medskip

{\bf Many were reluctant to use (D). In my experience, several excellent teachers I know use (D) quite frequently---I should practice it! Why might good teachers use (D) more? Does using (D) make you a better teacher?}


\es

\bs

{\bf Beyond dealing with misconduct, sometimes such issues can be avoided by changing the structure of the class. Perhaps teachers can run the class in ways that make cheating less possible.  Suggest one feature a class might have that encourages misconduct (but might have some other academic benefit) and another that discourages misconduct (but might have some other academic cost).}

\it

% 2014 ``Having students to submit their homework assignment to a plagiarism detection website is a method to discourage misconduct.''

{\it ``A feature that encourages misconduct is reusing previous exam or homework problems. There may be an academic benefit of reusing those problems if they are especially useful for teaching the course concepts. A feature that discourages misconduct is writing many versions of an exam that are randomly distributed to students. This may incur a cost if the exam questions need to be easier or less interesting in order to make so many versions.''}

\es

\bs 
{\it ``One way of encouraging misconduct is to more heavily weight homeworks (which are easy to cheat on) and less heavily weight exams. This makes cheating easier, but also likely has some benefits, as it ensures students focus on each homework, where much of the learning actually happens, especially for STEM classes.''

`` One way of discouraging misconduct would be to assign groups yourself during group assignments. This makes it harder for students to work with their friends (where is it more likely some students will carry the majority of the weight). However it does add another layer of work to the group assignment, so students may be less able to focus on the academic side as they deal with the communication side.''}

\es

\end{document}
